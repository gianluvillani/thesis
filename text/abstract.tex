

Working Memory (WM) is a general-purpose cognitive system responsible for temporary holding information in service of higher-order cognition systems. Such a system, capable of recalling and holding previous stored information, is of fundamental importance for complex tasks as reasoning, decision making and complex behaviour.

Artificial Neural Network models are powerful tools that often are used to \textit{bridge the gap between the cognitive perspective and the cortical perspective}, since they can model networks at scales that are currently hard or impossible to study \textit{in vivo}. According to the complexities and biological plausibility of the models, ANNs.  

In this thesis the dynamical properties of a special family of biologically inspired Artificial Neural Network called Bayesian Confidence Propagation Neural Network (BCPNN) is ana

    
    From a neuro-biologic point of view, early works based on monkey's neuronal activity while performing delayed tasks \cite{Fuster652} observed spiking activity in the Prefrontal Cortex (PFC) during the retention delays tasks. According to the first interpretation of the experimental data, this increased activity was due to persistent activity in the region under exam. 
    
    Nevertheless, recent works suggest \cite{Shafi2007VariabilityIN}, \cite{lundqvist2018working} that the traditional interpretation of spiking activity in the PFC can not be uniquely addressed in the Persistent Activity framework but some more complete investigation should be performed.
    
     In order to support this theory, recent works proposed alternative theoretical models (experimental evidence is still problematic), that view WM encoded by fast and volatile Hebbian synaptic plasticity and modulated non-Hebbian intrinsic excitability \cite{LansnerFRC}.
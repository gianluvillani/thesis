\subsubsection{Complete homogeneous graph}
In this section we consider a network of $N$ 4-dimensional oscillators whose interaction can be described by a complete, homogeneous undirected graph. In particular, we are interested on the conditions on the weight $w$ that ensures global synchronization. Again, it is particularly useful to simplify the dynamics in \cref{eq:complete_dynamics} and use instead \cref{eq:diff_dynamics}. 

The system's dynamics can be described as follows:

\begin{equation} 
\begin{aligned}
\dot d_{i} &= - d_i - e_i + k f(d_i) + w\sum\limits_{\substack{k=1 \\ k\neq i}}^N (f(d_k) - f(d_i)), \ w > 0\\
\tau_a \dot{e}_{i} &= g_a f(d_i) - e_{i} \\
\end{aligned}
\label{eq:diff_dynamics_network}
\end{equation}

Denote $D_k = d_1 - d_k, E_k = e_1 - e_k,\ \forall k \in \{2, \dots, N\}$. If one can show that $D_k, E_k,\ \forall k \in \{2, \dots, N\}$ go asymptotically to zero then the system achieves synchronisation, i.e. the system asymptotically converges to the manifold where $d_1 = d_2 = \dots = d_N$ and $e_1 = e_2 = \dots = e_N$. We can use Lyapunov stability analysis to find sufficient conditions on the coupling strength $\omega$ such that the system asymptotically converge to the synchronisation manifold.

Define the generic error dynamics:
\begin{equation} 
\begin{aligned}
\dot D_{k} &= - D_k - E_k - \alpha (f(d_1) - f(d_k)) \\
\tau_a \dot{E}_{k} &= g_a (f(d_1) - f(d_k)) - E_{k} \\
&(f(d_1) - f(d_k))D_k \geq 0
\end{aligned}
\label{eq:err_dynamics}
\end{equation}
and define $\alpha = \frac{1}{\tau_a}, \alpha <1 $. By multiplying the second equation in \cref{eq:err_dynamics} with $\alpha$, the error dynamics can be rewritten as:
\begin{equation} 
\begin{aligned}
\dot D_{k} &= - D_k - E_k - \alpha (f(d_1) - f(d_k)) \\
\dot{E}_{k} &=  \bar g (f(d_1) - f(d_k)) - \alpha E_{k}, \quad \bar g = \alpha g. \\
\end{aligned}
\label{eq:err_dynamics}
\end{equation}

Define the following Lyapunov function:
\begin{equation}
\begin{aligned}
V &= \sum\limits_{k=2}^N V_k, \\
\quad V_k &= \frac{1}{2}(\bar g D_k + \omega E_k)^2 + \frac{1}{2}\beta\omega D_k^2, \quad \beta = \bar g + (\alpha - 1)\omega \geq 0
\end{aligned}
\label{eq:Lyap_complete}
\end{equation}
By assuming $\omega< \frac{\alpha}{1-\alpha}g_a$ the function in \cref{eq:Lyap_complete} is indeed positive definite. In order to show that the candidate function $V$ is a proper Lyapunov function,  we have to find the conditions on the coupling strength $w$ such that $\dot V<0$. 
\begin{equation} 
\begin{aligned}
\dot V =  - \sum\limits_{k=2}^N & ((\bar gD_k + \omega E_k)^2 + \\
+ &\beta \bar g D_k E_k + \beta \omega E_k^2 + \omega \beta D_k^2 + \omega D_k E_k +\\
+ &\omega ^2 \beta D_k (f(d_1) - f(d_k)))
\end{aligned}
\label{eq:Lyap_complete_derivative}
\end{equation}
Rearranging the tems in \eqref{eq:Lyap_complete_derivative}:
\begin{equation} 
\begin{aligned}
\dot V =  - \sum\limits_{k=2}^N(& (\bar gD_k + \omega E_k)^2 + \\
+ &\omega \beta \left (E_k^2 + D_k^2 + \left(1 + \frac{\bar g}{\omega}\right) D_k E_k\right )  +\\
+ &\omega ^2 \beta D_k (f(d_1) - f(d_k)))
\end{aligned}
\label{eq:Lyap_complete_derivative_rearranged}
\end{equation}

$\omega \beta \left (E_k^2 + D_k^2 + \left(1 + \frac{\bar g}{\omega}\right) D_k E_k\right )$ is positive definite if $1 + \frac{\bar g}{\omega} <2$, i.e. $\omega > \alpha g$. Therefore, a sufficient condition for global synchronisation is:
\begin{equation} 
\begin{aligned}
\omega_{min} = g_a \alpha < \omega < \frac{\alpha}{1 - \alpha}g_a = \omega_{max}
\end{aligned}
\label{eq:omega_bounds}
\end{equation}

\subsubsection{Conditions for oscillations and synchronization}
As discussed in the bifurcation analysis section, for a homogeneous network, the dynamics of the system on the synchronisation manifold is affected both by the number of nodes $N$ and the magnitude of the coupling factor $\omega$: fixed $g_a, \tau_a$ the dynamics is shaped by the parameter $\alpha = (N-1)\omega$. We are interested in particular in the range $\left[2\left(1+\frac{1}{\tau_a}\right),\ g_a + 2\right] = [\alpha_{min}, \alpha_{max}]$\footnote{In the bifurcation analysis it was shown that the actual upper bound $\alpha_{max}$ is not exactly $g_a+2$ but a more complex function of the parameters $g_a, \tau_a$. For the case of $g_a =10, \tau_a=2$ this upper bound was computed numerically ($\alpha^{oo}\approx 13.11$). Therefore, for the sake of simplicity, we will use the more conservative upper bound $\alpha_{max}=g_a+2$}. In fact, for this range of values of $\alpha$ the oscillatory mode is the only stable mode (the only stable attractor is a limit cycle). Therefore, is interesting to find the range $[N_{min}, N_{max}]$ as a function of the parameters $g_a, \tau_a$ such that given the constraints in \cref{eq:omega_bounds}, $\alpha = (N-1)w \in [\alpha_{min}, \alpha_{max}]$:
\begin{equation} 
\begin{aligned}
\alpha_{min} < N \omega < \alpha_{max}, \quad \forall \omega \in [\omega_{min}, \omega_{max}], \quad N \in [N_{min}, N_{max}]
\end{aligned}
\label{eq:omega_alpha_bounds}
\end{equation}
The condition in \cref{eq:omega_alpha_bounds} is satisfied only if 
\begin{equation} 
\begin{aligned}
\min\limits_{\omega \in [\omega_{min}, \omega_{max}] }(N-1)\omega= (N-1)\omega_{min} \leq \alpha_{min} \\
\max\limits_{\omega \in [\omega_{min}, \omega_{max}] }(N-1)\omega = (N-1)\omega_{max} \leq \alpha_{max} 
\end{aligned}
\label{eq:omega_alpha_bounds}
\end{equation}
Therefore:
\begin{equation}
\frac{2}{g_a} \frac{1 + \alpha}{\alpha} + 1< N < \frac{1}{g_a} \frac{1 - \alpha}{\alpha}(g_a + 2) + 1
\label{eq:n_constraints}
\end{equation}
 


\subsubsection{Heterogeneous graph}
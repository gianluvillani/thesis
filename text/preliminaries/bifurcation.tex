
In this section we shortly introduce the concept of bifurcation and the different types of bifurcations that we will encounter later in the analysis. The following \textit{review (better)} is mostly based on the intuitive approach exposed in \cite{StrogatzNonlinear}. Although the bifurcation concept is general and could be applied to system with a higher number of dimensions, we will focus on reviewing bifurcation concepts for 2-dimensional systems. A more rigorous and complete analysis of bifurcation can be found in \cite{Guckenheimer1983}. 

Bifurcation theory provides the mathematical tools for investigating systems of physical interest with parameters which appear in the dynamical equations that define it. As these parameters varies, changes may occur in the qualitative behaviour of the solutions (equlibria are created or change their stability properties, limit cycles appears ...). These changes are called bifurcation and the parameter values are called bifurcation values. 

\begin{equation}
\dot{x}=f(x, \lambda), \quad x \in  \mathbb{R}  ^{n}, \quad  \lambda \in \mathbb{R}^{p}
\end{equation}

where $f$ is smooth. A bifurcation occurs at parameter $\lambda=\lambda_0$ if there are parameter values $\lambda_1$ arbitrarily close to $\lambda_0$ with dynamics topologically not equivalent from those at $\lambda_0$

\begin{itemize}
    \item How \textit{deep} should I go with the bifurcation math?
    \item Is a 'qualitative' approach enough?
    \item What about examples?
\end{itemize}

\subsection{Saddle-Node bifurcation}
The saddle-node bifurcation is the basic mechanism by which fixed points are created and destroyed. In a saddle-node bifurcation as the control parameter is varied, two fixed points collides and disappear. 
\subsection{Pitchfork bifurcation}
\subsubsection{Supercritical Pitchfork Bifurcation}
\subsubsection{Subcritical Pitchfork Bifurcation}
\subsection{Hopf bifurcation}
\subsubsection{Supercritical Hopf Bifurcation}
\subsubsection{Subcritical Hopf Bifurcation}